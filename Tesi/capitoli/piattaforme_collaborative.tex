\chapter{Piattaforme collaborative e di apprendimento} %\label{1cap:spinta_laterale}
% [titolo ridotto se non ci dovesse stare] {titolo completo}
%
\begin{citazione}
Questo capitolo fa una breve panoramica sulle piattaforme collaborative e di apprendimento considerate affini alla piattaforma \textsc{Code4Code}. Tale trattazione fornisce un'idea della motivazione per la quale si è ritenuto utile progettare la piattaforma e soprattutto formulare e risolvere problemi ad essa inerenti.
\end{citazione}
\newpage

\section{Differenza tra piattaforme collaborative e di apprendimento} %\label{1sec:scopo}
I concetti di piattaforma collaborativa e di piattaforma di apprendimento sono ben distinti. Una piattaforma collaborativa è, in generale, un tipo di piattaforma in cui gli utenti collaborano tra loro al fine di raggiungere uno scopo comune, mentre una piattaforma di apprendimento è un tipo di piattaforma che mira ad accrescere le conoscenze e le competenze di chi ne usufruisce. La natura del secondo tipo di piattaforma non è intrinsecamente legata ad un approccio basato sulla cooperazione tra gli utenti che vi partecipano, infatti nella maggior parte dei casi l'apprendimento risulta essere unidirezionale da insegnante ad allievo. Nell'ambito delle piattaforme di apprendimento vengono, spesso, inserite funzionalità di comunicazione tra allievo ed insegnante al fine di rendere agevole l'apprendimento tramite un confronto grazie al quale l'allievo può esporre i propri dubbi all'insegnante. Nonostante ciò, difficilmente il confronto tra le due parti avviene in tempo reale, e spesso si ricorre ad una sezione dedicata alle domande degli allievi a cui l'insegnante può rispondere in differita. Una piattaforma di collaborazione, d'altro canto, non è necessariamente legata all'apprendimento ma generalmente fornisce un insieme di strumenti per facilitare la comunicazione e la collaborazione in tempo reale.  
\section{Esempi di piattaforme collaborative}
Al giorno d'oggi esistono diverse piattaforme collaborative che consentono a chi ne usufruisce di collaborare insieme ad altri utenti per raggiungere uno scopo comune. Di seguito verranno elencate alcune piattaforme e ne verranno brevemente analizzate le caratteristiche fondamentali.
\begin{itemize}
    \item{\textbf{GitHub}: oltre a fungere da hosting per i progetti software degli utenti che ne usufruiscono, fornisce strumenti per la gestione del \emph{Version Control} che risulta essere di fondamentale importanza soprattutto nel caso di progetti di gruppo in cui gli sviluppatori devono collaborare tra loro.}
    \item{\textbf{Slack}: progettata per la collaborazione aziendale, fornisce un insieme di strumenti utili alla comunicazione istantanea tra gli utenti stessi. In particolare grazie a Slack è possibile avviare chiamate e videochiamate con funzionalità di condivisione schermo, usufruire di una chat in tempo reale e condividere file grazie all'integrazione con alcune piattaforme di condivisione come ad esempio \emph{Google Drive, Dropbox e Microsoft OneDrive}.}
\end{itemize}
\section{Esempi di piattaforme di apprendimento}
La progettazione di una piattaforma di apprendimento è un lavoro che determina in gran parte l'efficacia dell'apprendimento degli utenti che ne usufruiscono. Di seguito verranno esaminate alcune piattaforme con i relativi approcci alla progettazione dell'apprendimento.
\begin{itemize}    
    \item{\textbf{MasterClass}: mediante il pagamento di un abbonamento, consente agli utenti di accedere ad un insieme di corsi che abbracciano diversi ambiti, tra cui il business, la musica, la scrittura e la tecnologia. I corsi sono tenuti da esperti del settore, grazie ai quali la qualità dei corsi risulta essere elevata, tuttavia la piattaforma non garantisce un'interazione tra studenti ed insegnanti, sebbene non manchino alcune occasioni in cui gli studenti possono sottoporre il proprio lavoro ad una revisione da parte degli insegnanti.}
    \item{\textbf{Udemy}: consente agli utenti di usufruire di corsi (sia gratuiti che a pagamento) erogati da altri utenti della piattaforma. Ogni corso è suddiviso in lezioni, in corrispondenza di ognuna delle quali è indicato l'argomento affrontato mediante un breve titolo. È, inoltre, disponibile una sezione Q\&A in cui gli studenti del corso possono porre domande all'insegnate e chiunque può partecipare alla discussione. È disponibile una funzionalità di Messaggi Diretti tra studenti ed insegnanti solo per i corsi a pagamento al fine di ottenere delucidazioni sugli argomenti trattati nei corsi e per dare feedback. Infine gli insegnanti possono pubblicare degli annunci volti alla distribuzione di materiali gratuiti relativi ai loro corsi.}
\end{itemize}
\section{Tandem: una piattaforma che unisce collaborazione ed apprendimento}
La piattaforma \textsc{Code4Code} si ispira concettualmente ad una piattaforma che unisce collaborazione ed apprendimento: \textbf{Tandem}. Tandem è un'App sviluppata per Android e iOS che serve a mettere in contatto utenti che vogliono imparare nuove lingue. La piattaforma associa ad un utente che vuole imparare una lingua \emph{A} e conosce una lingua \emph{B} un utente che conosce la lingua \emph{A} e vuole imparare la lingua \emph{B}. I due utenti, una volta in contatto, possono dialogare effettuando un lavoro di mutua correzione, senza instaurare un rapporto insegnante-studente. Tale approccio è detto \emph{Tandem Language Learning} \cite{tandem} ed è il principio cardine della piattaforma \textsc{Code4Code}.\\Tandem fornisce come strumenti di supporto per la comunicazione chiamate, videochiamate e chat in tempo reale, in cui è possibile anche inviare messaggi vocali (fondamentali per esercitare la pronuncia della lingua). 
\section{Considerazioni sulle piattaforme analizzate}
Dalla breve analisi effettuata, risulta evidente che in molti casi ci sia un compromesso tra qualità dei corsi erogati ed interazione con gli insegnanti. L'approccio utilizzato da \emph{Tandem} fa emergere, in maniera evidente, tale compromesso, in quanto il generico interlocutore potrebbe non avere un background educazionale, per cui l'esperienza dal punto di vista pratico potrebbe risultare meno scorrevole di un'esperienza di insegnamento tradizionale come quella proposta da alcune delle piattaforme sopracitate in cui si partecipa a corsi pre-registrati. La problematica appena trattata rappresenta uno degli aspetti critici del metodo \emph{Tandem Language Learning}. D'altro canto tale approccio favorisce al massimo l'interazione tra le parti che vogliono apprendere e non rende necessario il pagamento di una somma di denaro, in quanto la moneta di scambio per la conoscenza è la conoscenza stessa. La piattaforma \textsc{Code4Code} propone un tipo di approccio all'apprendimento basato sulla collaborazione: se un utente \emph{X} conosce la Tecnologia \emph{A} e vuole studiare la Tecnologia \emph{B}, mentre un utente \emph{Y} conosce la Tecnologia \emph{B} e vuole studiare la Tecnologia \emph{A}, potranno mettersi in contatto e, tramite la piattaforma, scambiarsi conoscenze relative alle Tecnologie mediante lezioni in tempo reale tenute dagli utenti stessi, chat ed esercizi. 