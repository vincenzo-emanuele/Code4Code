\chapter{Conclusioni} %\label{1cap:spinta_laterale}
% [titolo ridotto se non ci dovesse stare] {titolo completo}
%


\begin{citazione}
Quest'ultimo capitolo conclude la trattazione del lavoro di tesi attraverso una riflessione su quanto sviluppato, ma soprattutto su quanto non è stato ancora sviluppato. Viene posta molta attenzione su quelli che saranno gli sviluppi futuri della piattaforma \textsc{Code4Code} e dei relativi strumenti di supporto che essa dovrà fornire, e vengono, inoltre, valutate diverse opzioni di fronte a scelte implementative future analizzando gli aspetti positivi e negativi di ognuna di esse. Pensare al futuro della piattaforma, infatti, è parte integrante del lavoro di tesi in quanto l'implementazione risulta essere parziale e lascia spazio ancora a tanto lavoro, sebbene gran parte della piattaforma sia stata ideata per intero. Per concludere, vengono effettuate considerazioni sulla generalizzabilità di quanto effettuato dal punto di vista dell'Intelligenza Artificiale; è stato, infatti, proposto un utilizzo alternativo dell'Agente Intelligente sviluppato, che non è contestualizzato nella piattaforma \textsc{Code4Code}.
\end{citazione}
\newpage

\section{Sviluppi futuri}
Come puntualizzato diverse volte nell'ambito della presente trattazione, il lavoro su cui ci si è soffermati maggiormente riguarda lo sviluppo dei moduli di Intelligenza Artificiale. Per tale ragione gli Agenti sviluppati sono stati testati in modo da verificare quanto sviluppato e, laddove sono stati rilevati risultati non esattamente conformi alla realtà, sono stati analizzati i motivi che hanno causato determinati problemi per comprendere se il risultato in questione fosse un semplice \emph{outlier} o se ci fossero gravi errori nell'impostazione dei parametri di addestramento del modello utilizzato. Diverse volte è risultato opportuno modificare radicalmente i DataSet utilizzati, mentre altre volte sono stati integrati quelli preesistenti con DataSet di tipologie diverse; tali tipi di operazioni, insieme alla correzione dei parametri di addestramento, hanno migliorato la qualità dei risultati ottenuti. Allo stato attuale, dunque, si dispone di un modulo di Intelligenza Artificiale completo e funzionante, testato con un discreto numero di tecnologie Software ed integrato in un'implementazione primordiale della piattaforma \textsc{Code4Code}. Tra gli sviluppi futuri sarà, sicuramente, prevista l'implementazione integrale della piattaforma \textsc{Code4Code} con tutti i relativi strumenti di supporto descritti nel capitolo precedente. Tale implementazione proseguirà, presumibilmente, sulla scia di quanto attualmente realizzato, anche dal punto di vista di tecnologie e Frameworks utilizzati. Per quanto attualmente implementato, infatti, l'utilizzo di \emph{Spring} ha portato alla realizzazione di un prototipo in maniera molto agevole e veloce, tuttavia vi sono numerose potenzialità di \emph{Spring} non ancora sfruttate, come ad esempio la possibilità di utilizzare \emph{JPA (Java Persistence API)} \cite{jpa} per la gestione dei dati persistenti della piattaforma. Sarà, inoltre, necessario prendere una decisione definitiva riguardante il sistema di comunicazione utilizzato dagli utenti: 
\begin{itemize}
	\item La prima opzione è l'utilizzo di \emph{Mattermost}: in tal caso sarebbe necessario apportare opportune modifiche al codice sorgente del Server sia lato \emph{Front-End} che lato \emph{Back-End} per evitare che gli utenti possano usufruire di \emph{Mattermost} in maniera impropria ed indipendente dalla piattaforma. Tale problema si pone in quanto \emph{Mattermost} è una piattaforma a sé stante e che prevede funzionalità che prescindono da quelle della piattaforma nella quale può, eventualmente, essere integrata. 
	\item La seconda opzione è la creazione di un sistema di comunicazione integrato ad-hoc per la piattaforma. Ciò non comporterebbe la risoluzione dei problemi di integrazione che si hanno nel caso di \emph{Mattermost}, ma sarebbe necessario riprogettare da zero un sistema di comunicazione in tempo reale che sia agevole nell'utilizzo e abbia tutte le caratteristiche e le funzionalità esposte in precedenza.
\end{itemize}
L'ultimo aspetto da trattare riguarda l'algoritmo di Intelligenza Artificiale che consiglia agli utenti dei partner con cui collaborare. Nel capitolo sull'Intelligenza Artificiale è stata proposta un'idea di progettazione ed implementazione dell'algoritmo, tuttavia affrontare concretamente un problema del genere richiede un lavoro non indifferente dal momento che è necessario scegliere, progettare ed implementare tutto ciò di cui un algoritmo genetico necessita per funzionare; ad esempio stabilire quale algoritmo adottare per gli operatori genetici che l'algoritmo dovrà utilizzare, può portare a grosse differenze nei risultati finali. 
\section{Decontestualizzazione del lavoro di Intelligenza Artificiale}
Sarebbe del tutto lecito chiedersi il motivo per il quale si sia deciso di sviluppare il modulo di Intelligenza Artificiale relativo al suggerimento di tecnologie Software, piuttosto che quello che suggerisce partner con cui collaborare. Il motivo di tale scelta risiede nel fatto che il modulo di Intelligenza Artificiale che consiglia tecnologie Software ha una duplice valenza, in quanto può essere decontestualizzato dalla piattaforma \textsc{Code4Code} ed utilizzato per altri scopi, mentre l'algoritmo che consiglia partner con cui collaborare è strettamente legato alla piattaforma e dipende da parametri ad essa relativi. L'algoritmo implementato può essere utilizzato al fine di effettuare studi statistici per comprendere quali Linguaggi e quali combinazioni di Frameworks e Librerie vengono spesso utilizzate insieme, il tutto in maniera assolutamente automatizzata basandosi su progetti reali ottenuti da \emph{GitHub}. Tali studi potrebbero essere utilizzati per molteplici scopi, come ad esempio il rilevamento di cambi di tendenza nel panorama di utilizzo di Frameworks e Librerie; a tale scopo bisognerebbe, chiaramente, prelevare le nuove Repositories periodicamente da \emph{GitHub} e assicurarsi di interpretare correttamente le variazioni che si ottengono nel tempo evitando di confondere differenze insignificanti con cambi di scenario che in realtà in quel momento non si stanno verificando.\\
Un altro tipo di applicazione dell'Agente implementato potrebbe consistere nel suggerimento dell'utilizzo di un determinato Framework o di una determinata Libreria nell'ambito di un progetto, sapendo che verranno impiegati determinati Linguaggi. Infatti, il DataSet di addestramento per il suggerimento di Frameworks è in larga parte composto da una lista di tag di Repositories, dalle quali è possibile evincere, con buona approssimazione, se in una determinata Repository venga utilizzato un certo Linguaggio insieme ad un certo Framework. La logica alla base di tale approccio risiede nel fatto che se un Linguaggio ed un Framework sono strettamente correlati tra loro, molto probabilmente sono stati utilizzati insieme un buon numero di volte, ragion per cui è lecito pensare che la scelta di quel determinato Framework sia sensata (a patto che il Framework da utilizzare assolva effettivamente allo scopo che si vuole conseguire). 
