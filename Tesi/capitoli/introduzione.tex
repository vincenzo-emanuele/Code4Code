\phantomsection
%\addcontentsline{toc}{chapter}{Introduzione}
\chapter{Introduzione}
\markboth{Introduzione}{}
% [titolo ridotto se non ci dovesse stare] {titolo completo}

\section{Contesto applicativo} %\label{1sec:scopo}
Il lavoro proposto nell'ambito della presente tesi si inserisce nel contesto applicativo delle piattaforme collaborative ed in particolar modo nell'ambito delle piattaforme che mirano all'acquisizione e al miglioramento delle competenze informatiche degli utenti ad esse iscritti mediante una collaborazione tra gli stessi. I benefici che possono essere ottenuti dalla collaborazione nell'ambito di Progetti di qualsiasi tipo, sono innumerevoli: si può pensare all'aumento della produttività, alla suddivisione del lavoro, alla possibilità di valutare diverse idee evitando di focalizzarsi solo su un numero ristretto di opzioni, alla crescita personale e a tanti altri vantaggi. Se poi si considera l'ambito dello sviluppo Software, ci si rende conto del fatto che la collaborazione non sia un optional in quanto, al giorno d'oggi, un qualsiasi tipo di Progetto Software richiede conoscenze di ambiti molto diversi fra loro, non solo dal punto di vista tecnologico. È certamente vero che nell'ambito di un Progetto Software sono presenti diversi Team che lavorano con tipi di Tecnologie tra loro diversi (ad esempio c'è chi si occupa del \emph{DataBase}, chi si occupa del \emph{Front-End}, eccetera), ma è altrettanto vero che la differenza tra le persone coinvolte in un Progetto Software risiede anche nel settore in cui sono specializzate, che non sempre è quello informatico. Dunque, anche se si volessero ignorare i vantaggi derivanti dalla collaborazione che riguardano la produttività e la suddivisione del lavoro, le competenze che sono richieste al giorno d'oggi per lavorare ad un Progetto sono difficilmente conciliabili in un unico individuo in quanto riguardano innumerevoli discipline. Risulta, dunque, di fondamentale importanza maturare un approccio alla collaborazione e soprattutto alla comunicazione in quanto non è scontato che persone aventi background diversi riescano a comunicare in maniera agevole ed efficiente. Collaborare insieme ad altre persone ad un Progetto non implica necessariamente riuscire a beneficiare di tutti i vantaggi della collaborazione: ci sono casi in cui scarse capacità di comunicazione hanno contribuito al fallimento di Progetti. Una recente ricerca \cite{SourceForge} finalizzata all'analisi delle motivazioni per le quali la piattaforma \emph{SourceForge} è stata abbandonata, ha evidenziato vari problemi derivanti da diversi fattori, tra cui i cosiddetti \emph{community smells} ossia 
\begin{quotation}
"caratteristiche organizzative e socio-tecniche sub-ottimali nell'ambito di una community software che può portare a costi aggiuntivi dovuti a problemi di comunicazione, rage-quitting e così via" (Tradotto)
\cite{CommunitySmells}   
\end{quotation}
sintomo di una forte instabilità dal punto di vista organizzativo e comunicativo. I community smells sono stati solo una delle cause che ha portato problemi alla piattaforma, tuttavia non vanno assolutamente sottovalutati in quanto sono comportamenti più comuni di quanto si possa pensare e possono portare a conseguenze peggiori di quelle previste.\\
Il lavoro di tesi svolto, porta il concetto di collaborazione nell'ambito dell'apprendimento di Tecnologie Software; tale iniziativa nasce dalla convinzione che anche in quest'ambito sia possibile sviluppare importanti capacità collaborative e soprattutto comunicative da poter sfruttare nell'ambito di Progetti Software. Al fine di contestualizzare il lavoro svolto è stato in parte progettato e sviluppato un prototipo di piattaforma collaborativa, denominato \textsc{Code4Code}, che mette a disposizione degli utenti un insieme di strumenti, tra cui l'Intelligenza Artificiale, che li assiste sia in fase di primo approccio alla piattaforma che in fasi più avanzate.  
\section{Obiettivi e risultati}
Sebbene l'idea iniziale del lavoro si focalizzasse sulla creazione completa della piattaforma \textsc{Code4Code}, nel corso dello sviluppo è stata posta una maggiore attenzione verso gli strumenti di supporto che essa dovrebbe fornire agli utenti che ne usufruiscono, in particolar modo alle tecniche di Intelligenza Artificiale utili a migliorare l'esperienza degli utenti. L'obiettivo finale del lavoro svolto è lo sviluppo di due micro-agenti di Intelligenza Artificiale che compongono un unico macro-agente che consiglia Linguaggi di Programmazione, Frameworks e Librerie da studiare sulla base di quelli conosciuti dall'utente che ne usufruisce. Il primo micro-agente consiglia Linguaggi di Programmazione a partire dai Linguaggi di Programmazione conosciuti dall'utente, mentre il secondo micro-agente si occupa di suggerire Frameworks e Librerie sulla base dei Linguaggi di Programmazione, dei Frameworks e delle Librerie conosciuti dall'utente. La suddivisione del macro-agente in due micro-agenti, nonostante la loro apparente similitudine, è stata dettata dalle differenze tra le tecniche di Intelligenza Artificiale impiegate nella realizzazione dei due micro-agenti: il primo micro-agente, infatti, sfrutta le \emph{Association Rules}, mentre il secondo si serve di tecniche di \emph{Natural Language Processing}. L'utilizzo dell'Intelligenza Artificiale verrà debitamente trattato nel corso dei successivi capitoli.\\Dopo aver esaminato il dominio del problema e stabilito quali tecniche di Intelligenza Artificiale utilizzare, sono stati studiati i fondamenti teorici di tali tecniche che hanno reso possibile, in fase implementativa, un utilizzo consapevole delle Librerie adoperate per la creazione degli Agenti Intelligenti di interesse. Durante la fase implementativa ci si è prima di tutto soffermati sulla creazione dei singoli micro-agenti, successivamente è stato effettuato un lavoro di integrazione tra i micro-agenti stessi ed infine è stata aggiunta un'interfaccia utente per rendere possibile un utilizzo agevole di quanto implementato.\\La fase implementativa ha richiesto numerose revisioni del DataSet e dei parametri di addestramento dei micro-agenti fino al raggiungimento di un risultato che, rapportato alle caratteristiche dei DataSet in esame e alla conoscenza dell'ambito delle attuali Tecnologie Software utilizzate, fosse intuitivamente sensato e contenesse il minor numero possibile di \emph{outliers}. Il risultato finale di tale lavoro di tesi risulta, dunque, essere un prototipo della piattaforma Web \textsc{Code4Code} tramite il quale si può interagire con l'Intelligenza Artificiale sviluppata.  
\section{Struttura della tesi}
La trattazione del lavoro di tesi verrà suddivisa in cinque capitoli di cui viene data, di seguito, una breve descrizione:
\begin{itemize}
    \item Il Primo Capitolo introduce il contesto applicativo, gli obiettivi e i risultati del lavoro di tesi svolto
    \item Il Secondo Capitolo analizza le caratteristiche fondamentali di alcune piattaforme collaborative considerate particolarmente rilevanti e affini alla piattaforma \textsc{Code4Code}
    \item Il Terzo Capitolo rappresenta il cuore della tesi in quanto descrive nel dettaglio le tecniche di Intelligenza Artificiale adoperate motivando le scelte che sono state effettuate a tal proposito
    \item Il Quarto Capitolo illustra la progettazione e l'architettura della piattaforma Web \textsc{Code4Code} presentando le tecnologie adoperate per l'effettiva creazione del prototipo implementato
    \item Il Quinto Capitolo conclude la trattazione del lavoro di tesi presentando considerazioni di carattere generale sul lavoro svolto e sugli eventuali sviluppi futuri
\end{itemize}