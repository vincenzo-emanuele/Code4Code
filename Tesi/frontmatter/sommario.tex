%\selectlanguage{italian}
\begin{abstract}

Il contesto applicativo della Tesi è incentrato sullo studio e sull'applicazione di metodologie di Intelligenza Artificiale nell'ambito di una piattaforma di collaborazione. L'obiettivo principale della tesi sviluppata è quello di fornire un insieme di strumenti di Intelligenza Artificiale da integrare in un'ipotetica piattaforma di collaborazione. Lo scopo di tale piattaforma è quello di mettere in contatto persone intenzionate ad imparare nuove tecnologie in ambito informatico, in modo che possano iniziare una collaborazione che consiste nello scambio di conoscenze relative alle tecnologie che desiderano apprendere. Lo sviluppo della Tesi non si concentra sulla creazione della piattaforma completa, bensì sullo studio, sul confronto, sulla progettazione e sull'implementazione di algoritmi di Intelligenza Artificiale che possano assistere gli utenti nell'utilizzo della piattaforma stessa. La Tesi fornisce, dunque, un prototipo denominato Code4Code che mostra il funzionamento degli agenti di Intelligenza Artificiale sviluppati. Il problema principale che è stato risolto mediante l'utilizzo dell'Intelligenza Artificiale è quello relativo al suggerimento di tecnologie da imparare sulla base di quelle già conosciute dall'utente; a tale scopo all'utente vengono consigliate sia tecnologie simili a quelle che già conosce, sia tecnologie spesso utilizzate con quelle che già conosce.
\\[1cm]
\end{abstract} 